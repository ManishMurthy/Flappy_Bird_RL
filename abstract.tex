\documentclass[12pt]{article}
\usepackage{geometry}
\usepackage{lipsum}

% Title of your project
\title{Play Flappy Bird Using Deep Reinforcement Learning}
\author{Manish Murthy, Sanjana Mandya Lokesh}
\date{}

\begin{document}

\maketitle

\begin{abstract}
Flappy Bird is a straightforward yet highly engaging game where players guide a bird, making it flap its wings to navigate through openings between pipes without crashing into them. This project investigates how a reinforcement learning (RL) agent can be trained to autonomously play Flappy Bird using Deep Q-Networks (DQN). Rather than manually coding rules, the agent learns through trial and error, determining when to flap and when to glide in order to survive for a longer duration.

The training process involves allowing the agent to interact with the game environment, where it monitors its position, speed, and the proximity to upcoming obstacles. Using these observations, a deep neural network predicts the optimal action to take. Initially, the agent plays with random actions, but over time it enhances its strategy by recalling previous experiences (experience replay) and making more informed decisions. It also employs an epsilon-greedy policy, which means it occasionally opts for random actions to explore new methods while gradually focusing on more intelligent decisions.

During training, the agent refines its choices by applying Q-learning, a method that updates its comprehension of which actions yield greater rewards (like staying alive longer). The primary objective is for the agent to maximize its score by avoiding crashes and manoeuvring through the course more skilfully. This project not only demonstrates the capability of AI to master games but also underscores fundamental reinforcement learning concepts that can be utilized in real-world scenarios, ranging from robotics to autonomous vehicles.

By the completion of the training, our AI-powered Flappy Bird agent illustrates how reinforcement learning can transform basic gameplay experiences into a polished strategy—without any human involvement apart from establishing the learning framework.
\end{abstract}

\begin{keywords}
Flappy Bird AI, Q-Learning, Reinforcement Learning (RL), Epsilon-Greedy Policy, Deep Q-Networks (DQN)
\end{keywords}

\end{document}